%% Created by Maple V Release 5 (IBM INTEL NT)
%% Source Worksheet: Untitled (1)
%% Generated: Sun Mar 06 01:44:44 2022
\documentclass{article}
\usepackage{maple2e}
\DefineParaStyle{Maple Output}
\DefineCharStyle{2D Math}
\DefineCharStyle{2D Output}
\begin{document}
\begin{maplegroup}
In this document we will describe how to build a generating function
for prime numbers.  

Prime numbers are not a well behaved sequence that could be generated
by a relatively 

simple generating function.  So expect the  level of complexity in
prime numbers 

to be reflected in this example.  

If we denote by (x, y) the edges of a rectangle then x*y is the
surface of that rectangle.

lets assume that each integer is a surface that could be represented
by one or more of 

these surfaces with the condition x and y belong to the set of
integers.

for example 7 is a prime number and it could be represented by (x, y)
= (7, 1) or (1, 7)

4 is not a prime and it could be represented by (4, 1)  (1, 4)  (2, 2)

4 is a full square so it will have the extra (2, 2) form.  Other
numbers

such as 8 could be represented by (4, 2)  (2, 4)  (8, 1) and (1, 8)

In this way every prime number have exactly 2 forms. every full square
have odd number

of forms,  the other numbers have even number of forms above 2.

First observe that the generating function

\end{maplegroup}
\begin{maplegroup}
\begin{mapleinput}
\mapleinline{active}{1d}{series(1/(1-x),x,10);}{%
}
\end{mapleinput}

\mapleresult
\begin{maplelatex}
\[
1 + x + x^{2} + x^{3} + x^{4} + x^{5} + x^{6} + x^{7} + x^{8} + x
^{9} + \mathrm{O}(x^{10})
\]
\end{maplelatex}

\end{maplegroup}
\begin{maplegroup}
is streght forward expansion whit x  powers  0,1,2....

\end{maplegroup}
\begin{maplegroup}
\begin{mapleinput}
\end{mapleinput}

\end{maplegroup}
\begin{maplegroup}
\begin{mapleinput}
\mapleinline{active}{1d}{series(1/(1-x^2),x,10);}{%
}
\end{mapleinput}

\mapleresult
\begin{maplelatex}
\[
1 + x^{2} + x^{4} + x^{6} + x^{8} + \mathrm{O}(x^{10})
\]
\end{maplelatex}

\end{maplegroup}
\begin{maplegroup}
The powers of the second series are 0,2,4,6... if we raise x to the
power n = 7  we get.

\end{maplegroup}
\begin{maplegroup}
\begin{mapleinput}
\mapleinline{active}{1d}{series(1/(1-x^7),x,20);}{%
}
\end{mapleinput}

\mapleresult
\begin{maplelatex}
\[
1 + x^{7} + x^{14} + \mathrm{O}(x^{21})
\]
\end{maplelatex}

\end{maplegroup}
\begin{maplegroup}
Now observe that the series of 1/(1-x\symbol{94}2) has 3 as the first
missing element which is a prime number.

we add 1/(1-x\symbol{94}2) to 1/(1-x\symbol{94}3) the next gap wich is
5 is a prime number. but we can do better by 

just adding all series and observe that all elements of the resulting
series starting with 2 are primes.

\end{maplegroup}
\begin{maplegroup}
\begin{mapleinput}
\mapleinline{active}{1d}{series(sum(1/(1-x^i),i = 1..6000),x,60);}{%
}
\end{mapleinput}

\mapleresult
\begin{maplelatex}
\begin{eqnarray*}
\lefteqn{6000 + x + 2\,x^{2} + 2\,x^{3} + 3\,x^{4} + 2\,x^{5} + 4
\,x^{6} + 2\,x^{7} + 4\,x^{8} + 3\,x^{9} + 4\,x^{10} + 2\,x^{11}
 + 6\,x^{12} + 2\,x^{13}} \\
 & &  + 4\,x^{14} + 4\,x^{15} + 5\,x^{16} + 2\,x^{17} + 6\,x^{18}
 + 2\,x^{19} + 6\,x^{20} + 4\,x^{21} + 4\,x^{22} + 2\,x^{23} + 8
\,x^{24} \\
 & &  + 3\,x^{25} + 4\,x^{26} + 4\,x^{27} + 6\,x^{28} + 2\,x^{29}
 + 8\,x^{30} + 2\,x^{31} + 6\,x^{32} + 4\,x^{33} + 4\,x^{34} + 4
\,x^{35} \\
 & &  + 9\,x^{36} + 2\,x^{37} + 4\,x^{38} + 4\,x^{39} + 8\,x^{40}
 + 2\,x^{41} + 8\,x^{42} + 2\,x^{43} + 6\,x^{44} + 6\,x^{45} + 4
\,x^{46} \\
 & &  + 2\,x^{47} + 10\,x^{48} + 3\,x^{49} + 6\,x^{50} + 4\,x^{51
} + 6\,x^{52} + 2\,x^{53} + 8\,x^{54} + 4\,x^{55} + 8\,x^{56} + 4
\,x^{57} \\
 & &  + 4\,x^{58} + 2\,x^{59} + \mathrm{O}(x^{60})
\mbox{\hspace{280pt}}
\end{eqnarray*}
\end{maplelatex}

\end{maplegroup}
\begin{maplegroup}
The example above limits the search to 6000 integers and the primes
within the set.  All prime powers

start with 2. Example 2x\symbol{94}41 means 41 is a prime.
5x\symbol{94}16 start with an odd number because it is a full square.

All other numbers start with even numbers that represent the number of
combinations of surfaces that could be

made from that number as explained above.

we can represent this frame by the function y = n/x where the graph of
this function passes by the top right 

corner of a rectangle x * y.  If n is a prime then the graph pass only
by (x, y) = (1, n) and (n, 1).

This method of finding primes by generating functin is costly in terms
of computing capacity.  The function

y = n/x could be used to derive algorithms that looks for primes
between two full square numbers of the 

approximity of a given number.

\end{maplegroup}
\end{document}
%% End of Maple V Output
